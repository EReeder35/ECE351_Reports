%%%%%%%%%%%%%%%%%%%%%%%%%%%%%%%%%%%%%%%%%%%%%%%%%%%%%%%%%%%%%%%%%
%                                                               %
%   Ethan Reeder                                                %
%   ECE351-53                                                   %
%   Lab 4 - System Step Response using Convolution              %
%   Due 16February2021                                          %
%   Comments: None                                              %
%                                                               %
%%%%%%%%%%%%%%%%%%%%%%%%%%%%%%%%%%%%%%%%%%%%%%%%%%%%%%%%%%%%%%%%%

\documentclass[12pt]{article}
\usepackage[letterpaper, portrait, margin=1in]{geometry}
\usepackage{listings}

\title{ECE 351 - Lab 4 - System Step Response using Convolution}
\author{Ethan Reeder}
\date{9 February 2021}

\usepackage{natbib}
\usepackage{graphicx}

\setlength{\parindent}{0pt}
\setlength{\parskip}{1em}

\begin{document}
\lstset{language=Python}

\maketitle

\newpage

\tableofcontents

\newpage

\section{Purpose}

The purpose of this lab was to get more familiar with convolution and how it can be used to find the step response of a system.

\section{Procedure}
This lab began by using the code for the step from Lab 2 to create the following functions in Python:

\begin{equation}
    h_1(t) = e^(2t) * u(1-t)
\end{equation}

\begin{equation}
    h_2(t) = u(t-2) - u(t-6)
\end{equation}

\begin{equation}
    h_3(t) = cos(\omega_0 t) * u(t)
\end{equation}

Given that $f_0 = 0.25 Hz$

These functions were then graphed, and the plots are in Figure 1 in the Results section.

Next, these functions were convolved with the step function to determine the step response using the convolution function created in Lab 3. The graphs for these responses are shown in Figures 2 through 4 in the Results section. Additionally, the equations for the response were derived from Equations 1 through 3 in the Equations section, and these were then graphed on the same plot as the computer-calculated convolutions to compare.

\section{Equations}

\begin{equation}
    y_1(t) = \int_{-\infty}^{\infty} [e^{2\tau} * u(1-\tau)] * u(t-\tau) \, d\tau
\end{equation}

\begin{equation}
    y_2(t) = \int_{-\infty}^{\infty} [u(\tau-2) - u(\tau-6)] * u(t-\tau) \, d\tau
\end{equation}

\begin{equation}
    y_3(t) = \int_{-\infty}^{\infty} [cos(\omega_0 \tau) * u(\tau)] * u(t-\tau) \, d\tau
\end{equation}

\section{Results}

\begin{figure}[h!]
    \centering
    \includegraphics[scale=.45]{part 1 task 2.png}
    \caption{Plots of h1, h2, and h3}
\end{figure}

\begin{equation}
    y_1(t) = \frac{1}{2} * e^{2t} * u(1-t) + e^{2} * u(t-1)
\end{equation}

\begin{equation}
    y_2(t) = (t-2) * u(t-2) - (t-6) * u(t-6)
\end{equation}

\begin{equation}
    y_3(t) = \frac{1}{\omega_0} * sin(\omega_0 t) * u(t)
\end{equation}

\begin{figure}[h!]
    \centering
    \includegraphics[scale=.35]{step response h1t.png}
    \caption{Step Response to h1(t)}
\end{figure}

\begin{figure}[h!]
    \centering
    \includegraphics[scale=.35]{step response h2t.png}
    \caption{Step Response to h2(t)}
\end{figure}

\begin{figure}[h!]
    \centering
    \includegraphics[scale=.35]{step response h3t.png}
    \caption{Step Response to h3(t)}
\end{figure}

\newpage

\section{Questions}

Leave any feedback on the clarity of lab tasks, expectations, and deliverables.

The lab was clear and the expectations were understood.

\section{GitHub Link}

https://github.com/EReeder35

\newpage

\section{Conclusion}

In conclusion, this lab was a useful expansion upon last lab. It shows how step responses work, and also solidified the connection between analytical (equation-based) convolution and the graphs and functions that are created in Python. One thing to note is that the graphs for the equation-based convolutions only go up to t=10, which is because that's the domain of t, and the convolution function uses the base of tExtended in order to calculate the convolution.

\end{document}