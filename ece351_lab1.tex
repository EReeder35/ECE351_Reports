%%%%%%%%%%%%%%%%%%%%%%%%%%%%%%%%%%%%%%%%%%%%%%%%%%%%%%%%%%%%%%%%%
%                                                               %
%   Ethan Reeder                                                %
%   ECE351-53                                                   %
%   Lab 1                                                       %
%   Due 26January2021                                           %
%   Comments: None                                              %
%                                                               %
%%%%%%%%%%%%%%%%%%%%%%%%%%%%%%%%%%%%%%%%%%%%%%%%%%%%%%%%%%%%%%%%%

\documentclass[12pt]{article}
\usepackage[letterpaper, portrait, margin=1in]{geometry}
\usepackage{listings}

\title{ECE 351 - Lab 1 - Introduction to Python and LaTex}
\author{Ethan Reeder}
\date{19 January 2021}

\usepackage{natbib}
\usepackage{graphicx}

\setlength{\parindent}{0pt}
\setlength{\parskip}{1em}

\begin{document}
\lstset{language=C}

\maketitle

\newpage

\tableofcontents

\newpage

\section{Objective}

The objective for this lab was to gain more knowledge with Spyder, Python, and LaTex.

This report will be a little different than the rest of the reports for this class. Instead of the normal sections (Introduction, Procedure, Equations, Questions, etc.), it will simply be an Introduction, a section for each portion of the lab procedure, with a Questions section at the end.

\section{Part 1 - Spyder IDE}

For this section, I read through the information in the link provided in the lab handout to become more familiar with the Spyder IDE, as well as the keyboard shortcut cheat sheet on BBLearn.

\section{Part 2 - Python Basics}

This section walked me through the basics of using Python to model variables, arrays, and matrices, as well as plots and complex numbers.

I worked through all the examples on my own, as shown in the .py file submitted with this report.

\section{Part 3 - LaTex Basics}

Finally, I went through the LaTex portion of the lab, which had a sample set of formatting commands. I've been using LaTex for all lab reports since Spring 2020, so this wasn't super new to me. I modified the format that I've been using to align more with the sample report on BBLearn, and reviewed my skills.

\section{Questions}

1. What course are you most excited for in your degree? What course have you enjoyed the most so far?

The course that I'm most excited for is my Digital Systems Engineering course (ECE440) because of how much I've enjoyed 240 and 340 so far. However, if I had to pick between the two, I'd say that 240 was my favorite class I've taken so far.

2. Leave any feedback on the clarity of the expectations, instructions, and deliverables.

Everything is clear and I understand what is expected of me in this lab.

\section{GitHub Link}

https://github.com/EReeder35


\end{document}